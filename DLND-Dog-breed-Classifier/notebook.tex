
% Default to the notebook output style

    


% Inherit from the specified cell style.




    
\documentclass[11pt]{article}

    
    
    \usepackage[T1]{fontenc}
    % Nicer default font (+ math font) than Computer Modern for most use cases
    \usepackage{mathpazo}

    % Basic figure setup, for now with no caption control since it's done
    % automatically by Pandoc (which extracts ![](path) syntax from Markdown).
    \usepackage{graphicx}
    % We will generate all images so they have a width \maxwidth. This means
    % that they will get their normal width if they fit onto the page, but
    % are scaled down if they would overflow the margins.
    \makeatletter
    \def\maxwidth{\ifdim\Gin@nat@width>\linewidth\linewidth
    \else\Gin@nat@width\fi}
    \makeatother
    \let\Oldincludegraphics\includegraphics
    % Set max figure width to be 80% of text width, for now hardcoded.
    \renewcommand{\includegraphics}[1]{\Oldincludegraphics[width=.8\maxwidth]{#1}}
    % Ensure that by default, figures have no caption (until we provide a
    % proper Figure object with a Caption API and a way to capture that
    % in the conversion process - todo).
    \usepackage{caption}
    \DeclareCaptionLabelFormat{nolabel}{}
    \captionsetup{labelformat=nolabel}

    \usepackage{adjustbox} % Used to constrain images to a maximum size 
    \usepackage{xcolor} % Allow colors to be defined
    \usepackage{enumerate} % Needed for markdown enumerations to work
    \usepackage{geometry} % Used to adjust the document margins
    \usepackage{amsmath} % Equations
    \usepackage{amssymb} % Equations
    \usepackage{textcomp} % defines textquotesingle
    % Hack from http://tex.stackexchange.com/a/47451/13684:
    \AtBeginDocument{%
        \def\PYZsq{\textquotesingle}% Upright quotes in Pygmentized code
    }
    \usepackage{upquote} % Upright quotes for verbatim code
    \usepackage{eurosym} % defines \euro
    \usepackage[mathletters]{ucs} % Extended unicode (utf-8) support
    \usepackage[utf8x]{inputenc} % Allow utf-8 characters in the tex document
    \usepackage{fancyvrb} % verbatim replacement that allows latex
    \usepackage{grffile} % extends the file name processing of package graphics 
                         % to support a larger range 
    % The hyperref package gives us a pdf with properly built
    % internal navigation ('pdf bookmarks' for the table of contents,
    % internal cross-reference links, web links for URLs, etc.)
    \usepackage{hyperref}
    \usepackage{longtable} % longtable support required by pandoc >1.10
    \usepackage{booktabs}  % table support for pandoc > 1.12.2
    \usepackage[inline]{enumitem} % IRkernel/repr support (it uses the enumerate* environment)
    \usepackage[normalem]{ulem} % ulem is needed to support strikethroughs (\sout)
                                % normalem makes italics be italics, not underlines
    

    
    
    % Colors for the hyperref package
    \definecolor{urlcolor}{rgb}{0,.145,.698}
    \definecolor{linkcolor}{rgb}{.71,0.21,0.01}
    \definecolor{citecolor}{rgb}{.12,.54,.11}

    % ANSI colors
    \definecolor{ansi-black}{HTML}{3E424D}
    \definecolor{ansi-black-intense}{HTML}{282C36}
    \definecolor{ansi-red}{HTML}{E75C58}
    \definecolor{ansi-red-intense}{HTML}{B22B31}
    \definecolor{ansi-green}{HTML}{00A250}
    \definecolor{ansi-green-intense}{HTML}{007427}
    \definecolor{ansi-yellow}{HTML}{DDB62B}
    \definecolor{ansi-yellow-intense}{HTML}{B27D12}
    \definecolor{ansi-blue}{HTML}{208FFB}
    \definecolor{ansi-blue-intense}{HTML}{0065CA}
    \definecolor{ansi-magenta}{HTML}{D160C4}
    \definecolor{ansi-magenta-intense}{HTML}{A03196}
    \definecolor{ansi-cyan}{HTML}{60C6C8}
    \definecolor{ansi-cyan-intense}{HTML}{258F8F}
    \definecolor{ansi-white}{HTML}{C5C1B4}
    \definecolor{ansi-white-intense}{HTML}{A1A6B2}

    % commands and environments needed by pandoc snippets
    % extracted from the output of `pandoc -s`
    \providecommand{\tightlist}{%
      \setlength{\itemsep}{0pt}\setlength{\parskip}{0pt}}
    \DefineVerbatimEnvironment{Highlighting}{Verbatim}{commandchars=\\\{\}}
    % Add ',fontsize=\small' for more characters per line
    \newenvironment{Shaded}{}{}
    \newcommand{\KeywordTok}[1]{\textcolor[rgb]{0.00,0.44,0.13}{\textbf{{#1}}}}
    \newcommand{\DataTypeTok}[1]{\textcolor[rgb]{0.56,0.13,0.00}{{#1}}}
    \newcommand{\DecValTok}[1]{\textcolor[rgb]{0.25,0.63,0.44}{{#1}}}
    \newcommand{\BaseNTok}[1]{\textcolor[rgb]{0.25,0.63,0.44}{{#1}}}
    \newcommand{\FloatTok}[1]{\textcolor[rgb]{0.25,0.63,0.44}{{#1}}}
    \newcommand{\CharTok}[1]{\textcolor[rgb]{0.25,0.44,0.63}{{#1}}}
    \newcommand{\StringTok}[1]{\textcolor[rgb]{0.25,0.44,0.63}{{#1}}}
    \newcommand{\CommentTok}[1]{\textcolor[rgb]{0.38,0.63,0.69}{\textit{{#1}}}}
    \newcommand{\OtherTok}[1]{\textcolor[rgb]{0.00,0.44,0.13}{{#1}}}
    \newcommand{\AlertTok}[1]{\textcolor[rgb]{1.00,0.00,0.00}{\textbf{{#1}}}}
    \newcommand{\FunctionTok}[1]{\textcolor[rgb]{0.02,0.16,0.49}{{#1}}}
    \newcommand{\RegionMarkerTok}[1]{{#1}}
    \newcommand{\ErrorTok}[1]{\textcolor[rgb]{1.00,0.00,0.00}{\textbf{{#1}}}}
    \newcommand{\NormalTok}[1]{{#1}}
    
    % Additional commands for more recent versions of Pandoc
    \newcommand{\ConstantTok}[1]{\textcolor[rgb]{0.53,0.00,0.00}{{#1}}}
    \newcommand{\SpecialCharTok}[1]{\textcolor[rgb]{0.25,0.44,0.63}{{#1}}}
    \newcommand{\VerbatimStringTok}[1]{\textcolor[rgb]{0.25,0.44,0.63}{{#1}}}
    \newcommand{\SpecialStringTok}[1]{\textcolor[rgb]{0.73,0.40,0.53}{{#1}}}
    \newcommand{\ImportTok}[1]{{#1}}
    \newcommand{\DocumentationTok}[1]{\textcolor[rgb]{0.73,0.13,0.13}{\textit{{#1}}}}
    \newcommand{\AnnotationTok}[1]{\textcolor[rgb]{0.38,0.63,0.69}{\textbf{\textit{{#1}}}}}
    \newcommand{\CommentVarTok}[1]{\textcolor[rgb]{0.38,0.63,0.69}{\textbf{\textit{{#1}}}}}
    \newcommand{\VariableTok}[1]{\textcolor[rgb]{0.10,0.09,0.49}{{#1}}}
    \newcommand{\ControlFlowTok}[1]{\textcolor[rgb]{0.00,0.44,0.13}{\textbf{{#1}}}}
    \newcommand{\OperatorTok}[1]{\textcolor[rgb]{0.40,0.40,0.40}{{#1}}}
    \newcommand{\BuiltInTok}[1]{{#1}}
    \newcommand{\ExtensionTok}[1]{{#1}}
    \newcommand{\PreprocessorTok}[1]{\textcolor[rgb]{0.74,0.48,0.00}{{#1}}}
    \newcommand{\AttributeTok}[1]{\textcolor[rgb]{0.49,0.56,0.16}{{#1}}}
    \newcommand{\InformationTok}[1]{\textcolor[rgb]{0.38,0.63,0.69}{\textbf{\textit{{#1}}}}}
    \newcommand{\WarningTok}[1]{\textcolor[rgb]{0.38,0.63,0.69}{\textbf{\textit{{#1}}}}}
    
    
    % Define a nice break command that doesn't care if a line doesn't already
    % exist.
    \def\br{\hspace*{\fill} \\* }
    % Math Jax compatability definitions
    \def\gt{>}
    \def\lt{<}
    % Document parameters
    \title{writeup}
    
    
    

    % Pygments definitions
    
\makeatletter
\def\PY@reset{\let\PY@it=\relax \let\PY@bf=\relax%
    \let\PY@ul=\relax \let\PY@tc=\relax%
    \let\PY@bc=\relax \let\PY@ff=\relax}
\def\PY@tok#1{\csname PY@tok@#1\endcsname}
\def\PY@toks#1+{\ifx\relax#1\empty\else%
    \PY@tok{#1}\expandafter\PY@toks\fi}
\def\PY@do#1{\PY@bc{\PY@tc{\PY@ul{%
    \PY@it{\PY@bf{\PY@ff{#1}}}}}}}
\def\PY#1#2{\PY@reset\PY@toks#1+\relax+\PY@do{#2}}

\expandafter\def\csname PY@tok@w\endcsname{\def\PY@tc##1{\textcolor[rgb]{0.73,0.73,0.73}{##1}}}
\expandafter\def\csname PY@tok@c\endcsname{\let\PY@it=\textit\def\PY@tc##1{\textcolor[rgb]{0.25,0.50,0.50}{##1}}}
\expandafter\def\csname PY@tok@cp\endcsname{\def\PY@tc##1{\textcolor[rgb]{0.74,0.48,0.00}{##1}}}
\expandafter\def\csname PY@tok@k\endcsname{\let\PY@bf=\textbf\def\PY@tc##1{\textcolor[rgb]{0.00,0.50,0.00}{##1}}}
\expandafter\def\csname PY@tok@kp\endcsname{\def\PY@tc##1{\textcolor[rgb]{0.00,0.50,0.00}{##1}}}
\expandafter\def\csname PY@tok@kt\endcsname{\def\PY@tc##1{\textcolor[rgb]{0.69,0.00,0.25}{##1}}}
\expandafter\def\csname PY@tok@o\endcsname{\def\PY@tc##1{\textcolor[rgb]{0.40,0.40,0.40}{##1}}}
\expandafter\def\csname PY@tok@ow\endcsname{\let\PY@bf=\textbf\def\PY@tc##1{\textcolor[rgb]{0.67,0.13,1.00}{##1}}}
\expandafter\def\csname PY@tok@nb\endcsname{\def\PY@tc##1{\textcolor[rgb]{0.00,0.50,0.00}{##1}}}
\expandafter\def\csname PY@tok@nf\endcsname{\def\PY@tc##1{\textcolor[rgb]{0.00,0.00,1.00}{##1}}}
\expandafter\def\csname PY@tok@nc\endcsname{\let\PY@bf=\textbf\def\PY@tc##1{\textcolor[rgb]{0.00,0.00,1.00}{##1}}}
\expandafter\def\csname PY@tok@nn\endcsname{\let\PY@bf=\textbf\def\PY@tc##1{\textcolor[rgb]{0.00,0.00,1.00}{##1}}}
\expandafter\def\csname PY@tok@ne\endcsname{\let\PY@bf=\textbf\def\PY@tc##1{\textcolor[rgb]{0.82,0.25,0.23}{##1}}}
\expandafter\def\csname PY@tok@nv\endcsname{\def\PY@tc##1{\textcolor[rgb]{0.10,0.09,0.49}{##1}}}
\expandafter\def\csname PY@tok@no\endcsname{\def\PY@tc##1{\textcolor[rgb]{0.53,0.00,0.00}{##1}}}
\expandafter\def\csname PY@tok@nl\endcsname{\def\PY@tc##1{\textcolor[rgb]{0.63,0.63,0.00}{##1}}}
\expandafter\def\csname PY@tok@ni\endcsname{\let\PY@bf=\textbf\def\PY@tc##1{\textcolor[rgb]{0.60,0.60,0.60}{##1}}}
\expandafter\def\csname PY@tok@na\endcsname{\def\PY@tc##1{\textcolor[rgb]{0.49,0.56,0.16}{##1}}}
\expandafter\def\csname PY@tok@nt\endcsname{\let\PY@bf=\textbf\def\PY@tc##1{\textcolor[rgb]{0.00,0.50,0.00}{##1}}}
\expandafter\def\csname PY@tok@nd\endcsname{\def\PY@tc##1{\textcolor[rgb]{0.67,0.13,1.00}{##1}}}
\expandafter\def\csname PY@tok@s\endcsname{\def\PY@tc##1{\textcolor[rgb]{0.73,0.13,0.13}{##1}}}
\expandafter\def\csname PY@tok@sd\endcsname{\let\PY@it=\textit\def\PY@tc##1{\textcolor[rgb]{0.73,0.13,0.13}{##1}}}
\expandafter\def\csname PY@tok@si\endcsname{\let\PY@bf=\textbf\def\PY@tc##1{\textcolor[rgb]{0.73,0.40,0.53}{##1}}}
\expandafter\def\csname PY@tok@se\endcsname{\let\PY@bf=\textbf\def\PY@tc##1{\textcolor[rgb]{0.73,0.40,0.13}{##1}}}
\expandafter\def\csname PY@tok@sr\endcsname{\def\PY@tc##1{\textcolor[rgb]{0.73,0.40,0.53}{##1}}}
\expandafter\def\csname PY@tok@ss\endcsname{\def\PY@tc##1{\textcolor[rgb]{0.10,0.09,0.49}{##1}}}
\expandafter\def\csname PY@tok@sx\endcsname{\def\PY@tc##1{\textcolor[rgb]{0.00,0.50,0.00}{##1}}}
\expandafter\def\csname PY@tok@m\endcsname{\def\PY@tc##1{\textcolor[rgb]{0.40,0.40,0.40}{##1}}}
\expandafter\def\csname PY@tok@gh\endcsname{\let\PY@bf=\textbf\def\PY@tc##1{\textcolor[rgb]{0.00,0.00,0.50}{##1}}}
\expandafter\def\csname PY@tok@gu\endcsname{\let\PY@bf=\textbf\def\PY@tc##1{\textcolor[rgb]{0.50,0.00,0.50}{##1}}}
\expandafter\def\csname PY@tok@gd\endcsname{\def\PY@tc##1{\textcolor[rgb]{0.63,0.00,0.00}{##1}}}
\expandafter\def\csname PY@tok@gi\endcsname{\def\PY@tc##1{\textcolor[rgb]{0.00,0.63,0.00}{##1}}}
\expandafter\def\csname PY@tok@gr\endcsname{\def\PY@tc##1{\textcolor[rgb]{1.00,0.00,0.00}{##1}}}
\expandafter\def\csname PY@tok@ge\endcsname{\let\PY@it=\textit}
\expandafter\def\csname PY@tok@gs\endcsname{\let\PY@bf=\textbf}
\expandafter\def\csname PY@tok@gp\endcsname{\let\PY@bf=\textbf\def\PY@tc##1{\textcolor[rgb]{0.00,0.00,0.50}{##1}}}
\expandafter\def\csname PY@tok@go\endcsname{\def\PY@tc##1{\textcolor[rgb]{0.53,0.53,0.53}{##1}}}
\expandafter\def\csname PY@tok@gt\endcsname{\def\PY@tc##1{\textcolor[rgb]{0.00,0.27,0.87}{##1}}}
\expandafter\def\csname PY@tok@err\endcsname{\def\PY@bc##1{\setlength{\fboxsep}{0pt}\fcolorbox[rgb]{1.00,0.00,0.00}{1,1,1}{\strut ##1}}}
\expandafter\def\csname PY@tok@kc\endcsname{\let\PY@bf=\textbf\def\PY@tc##1{\textcolor[rgb]{0.00,0.50,0.00}{##1}}}
\expandafter\def\csname PY@tok@kd\endcsname{\let\PY@bf=\textbf\def\PY@tc##1{\textcolor[rgb]{0.00,0.50,0.00}{##1}}}
\expandafter\def\csname PY@tok@kn\endcsname{\let\PY@bf=\textbf\def\PY@tc##1{\textcolor[rgb]{0.00,0.50,0.00}{##1}}}
\expandafter\def\csname PY@tok@kr\endcsname{\let\PY@bf=\textbf\def\PY@tc##1{\textcolor[rgb]{0.00,0.50,0.00}{##1}}}
\expandafter\def\csname PY@tok@bp\endcsname{\def\PY@tc##1{\textcolor[rgb]{0.00,0.50,0.00}{##1}}}
\expandafter\def\csname PY@tok@fm\endcsname{\def\PY@tc##1{\textcolor[rgb]{0.00,0.00,1.00}{##1}}}
\expandafter\def\csname PY@tok@vc\endcsname{\def\PY@tc##1{\textcolor[rgb]{0.10,0.09,0.49}{##1}}}
\expandafter\def\csname PY@tok@vg\endcsname{\def\PY@tc##1{\textcolor[rgb]{0.10,0.09,0.49}{##1}}}
\expandafter\def\csname PY@tok@vi\endcsname{\def\PY@tc##1{\textcolor[rgb]{0.10,0.09,0.49}{##1}}}
\expandafter\def\csname PY@tok@vm\endcsname{\def\PY@tc##1{\textcolor[rgb]{0.10,0.09,0.49}{##1}}}
\expandafter\def\csname PY@tok@sa\endcsname{\def\PY@tc##1{\textcolor[rgb]{0.73,0.13,0.13}{##1}}}
\expandafter\def\csname PY@tok@sb\endcsname{\def\PY@tc##1{\textcolor[rgb]{0.73,0.13,0.13}{##1}}}
\expandafter\def\csname PY@tok@sc\endcsname{\def\PY@tc##1{\textcolor[rgb]{0.73,0.13,0.13}{##1}}}
\expandafter\def\csname PY@tok@dl\endcsname{\def\PY@tc##1{\textcolor[rgb]{0.73,0.13,0.13}{##1}}}
\expandafter\def\csname PY@tok@s2\endcsname{\def\PY@tc##1{\textcolor[rgb]{0.73,0.13,0.13}{##1}}}
\expandafter\def\csname PY@tok@sh\endcsname{\def\PY@tc##1{\textcolor[rgb]{0.73,0.13,0.13}{##1}}}
\expandafter\def\csname PY@tok@s1\endcsname{\def\PY@tc##1{\textcolor[rgb]{0.73,0.13,0.13}{##1}}}
\expandafter\def\csname PY@tok@mb\endcsname{\def\PY@tc##1{\textcolor[rgb]{0.40,0.40,0.40}{##1}}}
\expandafter\def\csname PY@tok@mf\endcsname{\def\PY@tc##1{\textcolor[rgb]{0.40,0.40,0.40}{##1}}}
\expandafter\def\csname PY@tok@mh\endcsname{\def\PY@tc##1{\textcolor[rgb]{0.40,0.40,0.40}{##1}}}
\expandafter\def\csname PY@tok@mi\endcsname{\def\PY@tc##1{\textcolor[rgb]{0.40,0.40,0.40}{##1}}}
\expandafter\def\csname PY@tok@il\endcsname{\def\PY@tc##1{\textcolor[rgb]{0.40,0.40,0.40}{##1}}}
\expandafter\def\csname PY@tok@mo\endcsname{\def\PY@tc##1{\textcolor[rgb]{0.40,0.40,0.40}{##1}}}
\expandafter\def\csname PY@tok@ch\endcsname{\let\PY@it=\textit\def\PY@tc##1{\textcolor[rgb]{0.25,0.50,0.50}{##1}}}
\expandafter\def\csname PY@tok@cm\endcsname{\let\PY@it=\textit\def\PY@tc##1{\textcolor[rgb]{0.25,0.50,0.50}{##1}}}
\expandafter\def\csname PY@tok@cpf\endcsname{\let\PY@it=\textit\def\PY@tc##1{\textcolor[rgb]{0.25,0.50,0.50}{##1}}}
\expandafter\def\csname PY@tok@c1\endcsname{\let\PY@it=\textit\def\PY@tc##1{\textcolor[rgb]{0.25,0.50,0.50}{##1}}}
\expandafter\def\csname PY@tok@cs\endcsname{\let\PY@it=\textit\def\PY@tc##1{\textcolor[rgb]{0.25,0.50,0.50}{##1}}}

\def\PYZbs{\char`\\}
\def\PYZus{\char`\_}
\def\PYZob{\char`\{}
\def\PYZcb{\char`\}}
\def\PYZca{\char`\^}
\def\PYZam{\char`\&}
\def\PYZlt{\char`\<}
\def\PYZgt{\char`\>}
\def\PYZsh{\char`\#}
\def\PYZpc{\char`\%}
\def\PYZdl{\char`\$}
\def\PYZhy{\char`\-}
\def\PYZsq{\char`\'}
\def\PYZdq{\char`\"}
\def\PYZti{\char`\~}
% for compatibility with earlier versions
\def\PYZat{@}
\def\PYZlb{[}
\def\PYZrb{]}
\makeatother


    % Exact colors from NB
    \definecolor{incolor}{rgb}{0.0, 0.0, 0.5}
    \definecolor{outcolor}{rgb}{0.545, 0.0, 0.0}



    
    % Prevent overflowing lines due to hard-to-break entities
    \sloppy 
    % Setup hyperref package
    \hypersetup{
      breaklinks=true,  % so long urls are correctly broken across lines
      colorlinks=true,
      urlcolor=urlcolor,
      linkcolor=linkcolor,
      citecolor=citecolor,
      }
    % Slightly bigger margins than the latex defaults
    
    \geometry{verbose,tmargin=1in,bmargin=1in,lmargin=1in,rmargin=1in}
    
    

    \begin{document}
    
    
    \maketitle
    
    

    
    \subsection{Project: Write an Algorithm for a Dog Breed Identification
App}\label{project-write-an-algorithm-for-a-dog-breed-identification-app}

\subsubsection{The goals / steps of this project are the
following:}\label{the-goals-steps-of-this-project-are-the-following}

In this notebook, it is required to make the first steps towards
developing an algorithm that could be used as part of a mobile or web
app. At the end of this project, the finished code will accept any
user-supplied image as input and do detections: * If a dog is detected
in the image, it will provide an estimate of the dog's breed, therefore
it needs two classification: \textbf{\texttt{dog-vs-non\_dog}} , and
\textbf{\texttt{dog\ breed}} classification.\\
* If a human is detected, it will provide an estimate of the dog breed
that is most resembling, therefore a \textbf{\texttt{face-detector}}
model is needed.

\subsubsection{\texorpdfstring{\textbf{Steps:}}{Steps:}}\label{steps}

\textbf{1. Datasets import}: The data set is downloaded and put under
the same directory as dog\_app.ipynb is located. dog images are under
the \textbf{\texttt{dog\_images}} directory, while human face images is
under \textbf{\texttt{lfw}} directory With data been loaded, the
following steps are to define three detectors mentioned above:
\texttt{human\ face\ detector,\ dog\ detector\ and\ dog\ breed\ detector}.
For all those three detector, either specific computer vision algorithms
or deep neural networks can be used. If neural network, either creating
one from scratch or using transfer learning method. \textbf{2. Human
face detector definition}: in cell 2 and 3 human face detector based on
\textbf{\texttt{Haar\ feature-based\ cascade\ classifiers}}, this model
achieved \textbf{\texttt{98\%}} and \textbf{\texttt{18\%}} face
detection accuaracy on human face data set and dog data set respectively
\textbf{3. Dog detector definition}: from cell 6 t0 9, a dog detector
using transfer learning is defined based on \textbf{\texttt{VGG16}}.
This model achieved around \textbf{\texttt{93\%}} and
\textbf{\texttt{0\%}} dog detection accuaracy on dog data set and human
face data set respectively. The final model uses
\textbf{\texttt{inception\_v3}} instead, which showed better performance
as \textbf{\texttt{95\%}} dog detection accuracy \textbf{4. Dog breed
detector definition}, cell 14 defines a dog breed detector from scratch.
It has 3 convolution layers and 2 fully connected layers shown below:
\texttt{Net(\ \ \ \ \ \ \ (layer1):\ Sequential(\ \ \ \ \ \ \ \ \ (0):\ Conv2d(3,\ 16,\ kernel\_size=(3,\ 3),\ stride=(1,\ 1),\ padding=(1,\ 1))\ \ \ \ \ \ \ \ \ (1):\ BatchNorm2d(16,\ eps=1e-05,\ momentum=0.1,\ affine=True,\ track\_running\_stats=True)\ \ \ \ \ \ \ \ \ (2):\ ReLU()\ \ \ \ \ \ \ \ \ (3):\ MaxPool2d(kernel\_size=2,\ stride=2,\ padding=0,\ dilation=1,\ ceil\_mode=False)\ \ \ \ \ \ \ )\ \ \ \ \ \ \ (layer2):\ Sequential(\ \ \ \ \ \ \ \ \ (0):\ Conv2d(16,\ 32,\ kernel\_size=(3,\ 3),\ stride=(1,\ 1),\ padding=(1,\ 1))\ \ \ \ \ \ \ \ \ (1):\ BatchNorm2d(32,\ eps=1e-05,\ momentum=0.1,\ affine=True,\ track\_running\_stats=True)\ \ \ \ \ \ \ \ \ (2):\ ReLU()\ \ \ \ \ \ \ \ \ (3):\ MaxPool2d(kernel\_size=2,\ stride=2,\ padding=0,\ dilation=1,\ ceil\_mode=False)\ \ \ \ \ \ \ )\ \ \ \ \ \ \ (layer3):\ Sequential(\ \ \ \ \ \ \ \ \ (0):\ Conv2d(32,\ 64,\ kernel\_size=(3,\ 3),\ stride=(1,\ 1),\ padding=(1,\ 1))\ \ \ \ \ \ \ \ \ (1):\ BatchNorm2d(64,\ eps=1e-05,\ momentum=0.1,\ affine=True,\ track\_running\_stats=True)\ \ \ \ \ \ \ \ \ (2):\ ReLU()\ \ \ \ \ \ \ \ \ (3):\ MaxPool2d(kernel\_size=2,\ stride=2,\ padding=0,\ dilation=1,\ ceil\_mode=False)\ \ \ \ \ \ \ )\ \ \ \ \ \ \ (fc1):\ Linear(in\_features=50176,\ out\_features=200,\ bias=True)\ \ \ \ \ \ \ (fc2):\ Linear(in\_features=200,\ out\_features=133,\ bias=True)\ \ \ \ \ \ \ (dropout):\ Dropout(p=0.3)\ \ \ \ \ )}
\\
This model achieved around \textbf{\texttt{11\%}} dog breed detection
accuaracy on dog data set \textbf{5. Final model definition}, The final
model is based on \textbf{\texttt{inception\_v3}}. The main reasons to
choose this model is: a) size is smaller than VGG, and b) better
performance on imagenet data set.\\
In this model, dog-non-dog classification and dog breed predication are
combined into the same inception model by adding one extra fully
connected layer called \textbf{self.fc\_dog\_breed} at the output with
size (4086,133), in parallel with the original final fully connected
output layer \textbf{self.fc}. The detailed steps are shown below: -
Saved torchvision/models/inception.py into a local directory where
dog\_app.ipynb is located with name as \textbf{inception\_dog.py} - Load
pretrained parameters:

The following lines (line 30 to 41) from file \textbf{inception\_dog.py}
shows how to load all pretrained parameters except new
\textbf{self.fc\_dog\_breed} layer's

\begin{Shaded}
\begin{Highlighting}[]
\ControlFlowTok{if}\NormalTok{ pretrained:}
    \ControlFlowTok{if} \StringTok{'transform_input'} \KeywordTok{not} \KeywordTok{in}\NormalTok{ kwargs:}
\NormalTok{        kwargs[}\StringTok{'transform_input'}\NormalTok{] }\OperatorTok{=} \VariableTok{True}
\NormalTok{    model }\OperatorTok{=}\NormalTok{ Inception3(}\OperatorTok{**}\NormalTok{kwargs)}
\NormalTok{    pretrained_dict }\OperatorTok{=}\NormalTok{ model_zoo.load_url(model_urls[}\StringTok{'inception_v3_google'}\NormalTok{])}
\NormalTok{    model_dict }\OperatorTok{=}\NormalTok{ model.state_dict()}
    \ControlFlowTok{for}\NormalTok{ name, param }\KeywordTok{in}\NormalTok{ model_dict.items():}
        \ControlFlowTok{if}\NormalTok{ name }\KeywordTok{not} \KeywordTok{in}\NormalTok{ pretrained_dict:}
            \ControlFlowTok{continue}
\NormalTok{        param }\OperatorTok{=}\NormalTok{ pretrained_dict[name].data}
\NormalTok{        model_dict[name].copy_(param)}
    \ControlFlowTok{return}\NormalTok{ model}
\end{Highlighting}
\end{Shaded}

The following line (line 72) from \textbf{inception\_dog.py} shows how
extra layer self.fc\_dog\_breed is added into \textbf{init()} function

\begin{Shaded}
\begin{Highlighting}[]
\VariableTok{self}\NormalTok{.fc_dog_breed }\OperatorTok{=}\NormalTok{ nn.Linear(}\DecValTok{2048}\NormalTok{, dog_breed_classes)}
\end{Highlighting}
\end{Shaded}

The following lines (Line 135 to 143) in file \textbf{inception\_dog.py}
shows how the new layers is added in parrallel with the original output
layer

\begin{Shaded}
\begin{Highlighting}[]
\CommentTok{# 1 x 1 x 2048}
\NormalTok{x }\OperatorTok{=}\NormalTok{ x.view(x.size(}\DecValTok{0}\NormalTok{), }\OperatorTok{-}\DecValTok{1}\NormalTok{)}
\NormalTok{x_dog_breed }\OperatorTok{=} \VariableTok{self}\NormalTok{.fc_dog_breed(x)           }
\CommentTok{# 2048}
\NormalTok{x }\OperatorTok{=} \VariableTok{self}\NormalTok{.fc(x)   }
\CommentTok{# 1000 (num_classes)}
\ControlFlowTok{if} \VariableTok{self}\NormalTok{.training }\KeywordTok{and} \VariableTok{self}\NormalTok{.aux_logits:}
    \ControlFlowTok{return}\NormalTok{ x_dog_breed, x, aux }
\ControlFlowTok{return}\NormalTok{ x_dog_breed, x}
\end{Highlighting}
\end{Shaded}

During training, all layers parameters are freezed except added
\textbf{self.fc\_dog\_breed.weight} and
\textbf{self.fc\_dog\_breed.bias}. The batch normalization layers
running\_mean and running\_var are also freezed in cell 16 shown below:
\texttt{python\ if\ pretrained:\ \ \ \ \ for\ module\ in\ model.modules():\ \ \ \ \ \ \ \ \ if\ isinstance(module,\ torch.nn.modules.BatchNorm1d):\ \ \ \ \ \ \ \ \ \ \ \ \ module.eval()\ \ \ \ \ \ \ \ \ if\ isinstance(module,\ torch.nn.modules.BatchNorm2d):\ \ \ \ \ \ \ \ \ \ \ \ \ module.eval()\ \ \ \ \ \ \ \ \ if\ isinstance(module,\ torch.nn.modules.BatchNorm3d):\ \ \ \ \ \ \ \ \ \ \ \ \ module.eval()}
\textbf{If not freezing batch normalization layer's running\_mean and
running\_var, these values will be updated when training model for
dog-breed classes. These updated running\_mean and running\_var will
reduce model accuracy when doing dog-non-dog classification from 95\% to
81\%} Final model's dog breed prediction accuracy is up to \textbf{85\%}
after 5 epoch training, while dog-non-dog classification accuracy stays
the same as before training, \textbf{95\%}

\#\#\# \textbf{Results:} \#\#\#\# Performance on dog\_images Performance
is checked on five images from test dataset, two images got from
intenet, and three skatched dog pictures The following five pictures
show correct dog breed predictions on the test data

\begin{verbatim}
<figcaption>American water spaniel</figcaption>
<img  src="dog_detected/American water spaniel.jpg" alt="Drawing" style="width: 600px;"/>
<figcaption>Brittany</figcaption>
<img  src="dog_detected/Brittany.jpg" alt="Drawing" style="width: 600px;"/>
<figcaption>Curly-coated retriever</figcaption>
<img  src="dog_detected/Curly-coated retriever.jpg" alt="Drawing" style="width: 600px;"/> 
<figcaption>Labrador retriever</figcaption>
<img  src="dog_detected/Labrador retriever.jpg" alt="Drawing" style="width: 600px;"/> 
<figcaption>Welsh springer spaniel</figcaption>
<img  src="dog_detected/Welsh springer spaniel.jpg" alt="Drawing" style="width: 600px;"/>     
</figure>  
\end{verbatim}

The following two pictures show correct dog breed predictions on
pictures got from \textbf{internet}

\begin{verbatim}
<figcaption>beagle puppy</figcaption>
<img  src="dog_detected/beagle puppy (from internet).jpg" alt="Drawing" style="width: 600px;"/>
<figcaption>german shepherd</figcaption>
<img  src="dog_detected/german shepherd (from internet).jpg" alt="Drawing" style="width: 600px;"/>
</figure>    
\end{verbatim}

The following three pictures show correct dog breed predictions on
\textbf{skatched} dog pictures

\begin{verbatim}
<figcaption>dalmatian breed</figcaption>
<img  src="dog_detected/dalmatian breed.jpg" alt="Drawing" style="width: 600px;"/>
<figcaption>Dog</figcaption>
<img  src="dog_detected/Dog.jpg" alt="Drawing" style="width: 600px;"/>
<figcaption>labrador Retriever </figcaption>
<img  src="dog_detected/labrador Retriever .jpg" alt="Drawing" style="width: 600px;"/>    
</figure>  
\end{verbatim}

\paragraph{Performance on human\_face}\label{performance-on-human_face}

Performance is checked on two human face images

\begin{verbatim}
<figcaption>Aaron Tippin</figcaption>
<img  src="human_face_detected/Aaron Tippin.jpg" alt="Drawing" style="width: 600px;"/>
<figcaption>Aileen Riggin Soule</figcaption>
<img  src="human_face_detected/Aileen Riggin Soule.jpg" alt="Drawing" style="width: 600px;"
\end{verbatim}

style="width: 600px;"/\textgreater{}\\

\paragraph{Performance on non human\_face or dog
images}\label{performance-on-non-human_face-or-dog-images}

Performance is checked on five images: 1 car, two cats and two wolves

\begin{verbatim}
<figcaption>car</figcaption>
<img  src="no_human_face/car.jpg" alt="Drawing" style="width: 600px;"/>
<figcaption>cat</figcaption>
<img  src="no_human_face/cat.jpg" alt="Drawing" style="width: 600px;"
\end{verbatim}

style="width: 600px;"/\textgreater{}\\

Cute\_Cat

\begin{verbatim}
<img  src="no_human_face/Cute Cat.jpg" alt="Drawing" style="width: 600px;"/>
<figcaption>wolf</figcaption>
<img  src="no_human_face/wolf.jpg" alt="Drawing" style="width: 600px;"
\end{verbatim}

style="width: 600px;"/\textgreater{}\\

wolves gray

\begin{verbatim}
<img  src="no_human_face/wolves gray.jpg" alt="Drawing" style="width: 600px;"/>    
</figure>   
\end{verbatim}


    % Add a bibliography block to the postdoc
    
    
    
    \end{document}
